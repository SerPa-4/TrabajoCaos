
\documentclass[11pt]{article}


\usepackage[spanish]{babel} %Acentos 
\renewcommand\spanishtablename{Tabla} %Cambiar Cuadro por Tabla
\usepackage{array} %Cambiar grosor de lineas en las tablas c!{\vrule width 1pt} para columnas \ChangeRT{1pt} para filas en vez de \hline
\newcommand\ChangeRT[1]{\noalign{\hrule height #1}}
\usepackage{amsmath}
\usepackage{amsfonts}
\usepackage{dsfont}
\usepackage{graphicx}
\usepackage{geometry}
\usepackage{subfig}
\usepackage{amsthm}
\usepackage{enumerate} %listas 
\usepackage{multirow} %combinar filas tablas
\usepackage{setspace} %Margenes y demas
\usepackage[symbol]{footmisc} %footnotes
\usepackage[table,xcdraw,dvipsnames]{xcolor} %color de tablas
\usepackage{listings} %paquete para poner el codigo
\usepackage{verbatim} %Comentarios de varias lineas \begin{comment}
\usepackage{vmargin} %para margenes de la hoja y demas
\usepackage{hyperref} %clicks en las referencias
	\hypersetup{
    	colorlinks,
   		citecolor=blue,
    	filecolor=blue,
    	linkcolor=blue,
    	urlcolor=blue
	}

\begin{document}
\begin{titlepage}
\centering
%{\includegraphics[width=0.5\textwidth]{fisicateorica}\par}
\vspace{1cm}
{\bfseries\LARGE Universidad de Zaragoza \par}
\vspace{1cm}
{\scshape\Large Área de Física de la Materia Condensada\\
-\\
Facultad de Ciencias \par}
\vspace{2cm}
{\scshape\Huge Trabajo de Caos y Sistemas Dinámicos no lineales \par}
\vspace{0.5cm}
{\scshape\Large Trabajo número 11:\\
Guerra de Mafias \par}
\vspace{0.3cm}
Tellería Serrano, Oriol\footnote[2]{777777@unizar.es}
Pardina Quirós, Sergio\footnote[3]{761224@unizar.es}
Muro Belloso, Alejandro\footnote[4]{759731@unizar.es} \par
\vspace{1cm}
%{\scshape\Large Grupo A1 \par}
\vspace{1cm}
%{\itshape\LARGE Técnicas Físicas II \par}
%\vspace{3cm}
{\Large 12 - 05 - 2020 \par}
\end{titlepage}

\spacing{1.2}
%\setpapersize{A4}
\begin{comment}
\setmargins{2.5cm}
{1.5cm}                        % margen superior
{16.5cm}                      % anchura del texto
{23.42cm}                    % altura del texto
{10pt}                           % altura de los encabezados
{1cm}                           % espacio entre el texto y los encabezados
{0pt}                             % altura del pie de página
{2cm}    % espacio entre el texto y el pie de página
\end{comment}
\newpage

\section*{Análisis del sistema dinámico bidimensional}

La descripción analítica exacta del comportamiento de los agentes viene dada, en el límite de una población bien mezclada, por la \textit{Ecuación Replicador} bidimensional dada por:
\begin{equation}\label{ec1}
\dot{x}_1=x_1(1-x_1)\lbrace(N_1-1)[x_1(1-b+r)-r]+N_2p[x_2(1-b+\epsilon)-\epsilon]\rbrace
\end{equation}
\begin{equation}\label{ec2}
\dot{x}_2=x_2(1-x_2)\lbrace(N_2-1)[x_2(1-b+r)-r]+N_1p[x_1(1-b+\epsilon)-\epsilon]\rbrace
\end{equation}
Donde $X_\alpha$ representa la fracción de cooperadores en la población $\alpha$ (=1,2).\\
En el límite termodinámico cunado $N_1,N_2\rightarrow\infty$ tenemos:
\begin{equation}\label{ec3}
\dot{x}_1=x_1(1-x_1)\lbrace N_1[x_1(1-b+r)-r]+N_2p[x_2(1-b+\epsilon)-\epsilon]\rbrace
\end{equation}
\begin{equation}\label{ec4}
\dot{x}_2=x_2(1-x_2)\lbrace N_2[x_2(1-b+r)-r]+N_1p[x_1(1-b+\epsilon)-\epsilon]\rbrace
\end{equation}\\




\noindent Comenzamos el análisis obteniendo las nulclinas del sistema, aquellas curvas que corresponden a $\dot{x}_1=\dot{x}_2=0$:

\subsection*{Igualando la \autoref{ec3} a 0 tenemos:}
\begin{equation*}
\dot{x}_1=0=\underline{x_1(1-x_1)}\lbrace N_1[x_1(1-b+r)-r]+N_2p[x_2(1-b+\epsilon)-\epsilon]\rbrace
\end{equation*}
Fijándonos en la parte subrayada obtenemos inmediatamente las soluciones:
\begin{equation}\label{ec5}
x_1=0
\end{equation}
\begin{equation}\label{ec6}
x_1=1
\end{equation}
La última solución se obtiene eliminando la parte subrayada, no  negativa pues no consideramos $x_1=0,1$, y con $\beta=N_1/N_2\geq0$:
\begin{equation*}
\beta[x_1(1-b+r)-r]+p[x_2(1-b+\epsilon)-\epsilon]=0\Longleftrightarrow
\end{equation*}
\begin{equation}\label{ec7}
\Longleftrightarrow x_2=\dfrac{-x_1\beta(b-1-r)-(\beta r+p\epsilon)}{p(b-1-\epsilon)}
\end{equation}

\newpage

\subsection*{Igualando la \autoref{ec4} a 0 tenemos:}
\begin{equation*}
\dot{x}_2=0=\underline{x_2(1-x_2)}\lbrace N_2[x_2(1-b+r)-r]+N_1p[x_1(1-b+\epsilon)-\epsilon]\rbrace
\end{equation*}
Fijándonos en la parte subrayada obtenemos inmediatamente las soluciones:
\begin{equation}\label{ec8}
x_2=0
\end{equation}
\begin{equation}\label{ec9}
x_2=1
\end{equation}
La última solución se obtiene eliminando la parte subrayada, no  negativa pues no consideramos $x_2=0,1$, y con $\beta=N_1/N_2\geq0$:
\begin{equation*}
[x_2(1-b+r)-r]+\beta p[x_1(1-b+\epsilon)-\epsilon]=0\Longleftrightarrow
\end{equation*}
\begin{equation}\label{ec10}
\Longleftrightarrow x_2=\dfrac{-x_1\beta p(b-1-\epsilon)-(r+\beta p\epsilon)}{(b-1-r)}
\end{equation}\\

Juntando todo obtenemos los punto fijos en el plano $x_1,x_2$:

\subsubsection*{Con la \autoref{ec5} y la \autoref{ec8} obtenemos:}
\vspace{-0.7cm}
\begin{equation}\label{ec11}
x_1=0
\end{equation}
\begin{equation}\label{ec12}
x_2=0
\end{equation}

\subsubsection*{Con la \autoref{ec5} y la \autoref{ec9} obtenemos:}
\vspace{-0.7cm}
\begin{equation}\label{ec13}
x_1=0
\end{equation}
\begin{equation}\label{ec14}
x_2=1
\end{equation}



\subsubsection*{Con la \autoref{ec5} y la \autoref{ec10} obtenemos:}
\vspace{-0.7cm}
\begin{equation}\label{ec15}
x_1=0
\end{equation}
\begin{equation}\label{ec16}
x_2=-\dfrac{r+\beta p\epsilon}{b-1-r}
\end{equation}


\subsubsection*{Con la \autoref{ec6} y la \autoref{ec8} obtenemos:}
\vspace{-0.7cm}
\begin{equation}\label{ec17}
x_1=1
\end{equation}
\begin{equation}\label{ec18}
x_2=0
\end{equation}

\subsubsection*{Con la \autoref{ec6} y la \autoref{ec9} obtenemos:}
\vspace{-0.7cm}
\begin{equation}\label{ec19}
x_1=1
\end{equation}
\begin{equation}\label{ec20}
x_2=1
\end{equation}

\subsubsection*{Con la \autoref{ec6} y la \autoref{ec10} obtenemos:}
\vspace{-0.7cm}
\begin{equation}\label{ec21}
x_1=1
\end{equation}
\begin{equation}\label{ec22}
x_2=-\dfrac{\beta p(b-1)+r}{b-1-r})=-\beta p-r\dfrac{\beta p+1}{b-1-r}
\end{equation}


\subsubsection*{Con la \autoref{ec7} y la \autoref{ec8} obtenemos:}
\vspace{-0.5cm}
\begin{equation}\label{ec23}
x_1=-\dfrac{\beta r+p\epsilon}{\beta(b-1-r)}
\end{equation}
\begin{equation}\label{ec24}
x_2=0
\end{equation}

\subsubsection*{Con la \autoref{ec7} y la \autoref{ec9} obtenemos:}
\vspace{-0.5cm}
\begin{equation}\label{ec25}
x_1=-\dfrac{p(b-1)+\beta r}{\beta(b-1-r)}=-\dfrac{p}{\beta}-r\dfrac{p+\beta}{\beta(b-1-r)}
\end{equation}
\begin{equation}\label{ec26}
x_2=1
\end{equation}


\subsubsection*{Con la \autoref{ec7} y la \autoref{ec10} obtenemos:}
\vspace{-0.5cm}
\begin{equation*}
\dfrac{-x_1\beta(b-1-r)-(\beta r+p\epsilon)}{p(b-1-\epsilon)}=\dfrac{-x_1\beta p(b-1-\epsilon)-(r+\beta p\epsilon)}{(b-1-r)}\Longleftrightarrow
\end{equation*}
\begin{equation*}
\Longleftrightarrow x_1\beta[(p(b-1-\epsilon))^2-(b-1-r)^2]=(\beta r+p\epsilon)(b-1-r)-p(r+\beta p\epsilon)(b-1-\epsilon)\Longleftrightarrow
\end{equation*}
\begin{equation}\label{ec27}
\Longleftrightarrow x_1=\dfrac{(\beta r+p\epsilon)(b-1-r)-p(r+\beta p\epsilon)(b-1-\epsilon)}{\beta[(p(b-1-\epsilon))^2-(b-1-r)^2]}
\end{equation}\\
\begin{equation*}
x_2=\dfrac{-\beta p(b-1-\epsilon)}{(b-1-r)}\left(\dfrac{(\beta r+p\epsilon)(b-1-r)-p(r+\beta p\epsilon)(b-1-\epsilon)}{\beta[(p(b-1-\epsilon))^2-(b-1-r)^2]}\right)-\dfrac{(r+\beta p\epsilon)}{(b-1-r)}=
\end{equation*}
\begin{equation*}
-\dfrac{p(\beta r+p\epsilon)(b-1-\epsilon)}{[(p(b-1-\epsilon))^2-(b-1-r)^2]}+\dfrac{(p(b-1-\epsilon))^2(r+\beta p\epsilon)}{[(p(b-1-\epsilon))^2-(b-1-r)^2](b-1-r)}+
\end{equation*}
\begin{equation*}
\dfrac{-(r+\beta p\epsilon)(p(b-1-\epsilon))^2+(r+\beta p\epsilon)(b-1-r)^2]}{[(p(b-1-\epsilon))^2-(b-1-r)^2](b-1-r)}
\end{equation*}
\begin{equation}\label{ec28}
x_2=\dfrac{(r+\beta p\epsilon)(b-1-r)-p(\beta r+p\epsilon)(b-1-\epsilon)}{[(p(b-1-\epsilon))^2-(b-1-r)^2]}
\end{equation}

\newpage

\noindent Puesto que solo nos interesan los puntos dentro del cuadrado unidad, $0\leq x_1,x_2\leq 1$ debemos descartar los puntos que se hallen fuera de él.

Esto incluye los puntos dados por la \autoref{ec22}:\\
El denominador muestra una asíntota en $b=1+r$ con lo que distinguimos dos casos:\\
$1<b<1+r$: el denominador es negativo y el segundo término es positivo.
\begin{equation*}
x_2>x_2(b=1)=-\beta p+(\beta p+1)=1
\end{equation*}
$b>1+r$: el denominador es ahora positivo y el segundo término es negativo. Se tiene que en el límite:
\begin{equation*}
x_2(b\rightarrow\infty)=-\beta p<0
\end{equation*}


Y aquellos dados por la \autoref{ec25}:\\
El denominador muestra una asíntota en $b=1+r$ con lo que distinguimos dos casos:\\
$1<b<1+r$: el denominador es negativo y el segundo término es positivo.
\begin{equation*}
x_1>x_1(b=1)=-\dfrac{p}{\beta}+\left(\dfrac{p}{\beta}+1\right)=1
\end{equation*}
$b>1+r$: el denominador es ahora positivo y el segundo término es negativo. Se tiene que en el límite:
\begin{equation*}
x_1(b\rightarrow\infty)=-\dfrac{p}{\beta}<0
\end{equation*}\\

En conclusión quedan los siguientes puntos fijos:
\begin{equation*}
A=(0,1)
\end{equation*}
\begin{equation*}
B=(1,0)
\end{equation*}
\begin{equation*}
C=(1,1)
\end{equation*}
\begin{equation*}
D=(0,0)
\end{equation*}
\begin{equation*}
A'=\left(1,-\dfrac{\beta p(b-1)+r}{\beta(b-1-r)}\right)
\end{equation*}
\begin{equation*}
B'=\left(-\dfrac{p(b-1)+\beta r}{\beta(b-1-r)},1\right)
\end{equation*}
\begin{equation*}
E'=\left(x_1,x_2\right)
\end{equation*}
\vspace{-1.2cm}
\center{Dados por la \autoref{ec27} y la \autoref{ec28} respectivamente.}

\newpage

\noindent El caso de coincidencia de nulclinas se da cuando las rectas dadas por la \autoref{ec7} y la \autoref{ec10} son iguales, es decir tienen igual pendiente $m$ e igual ordenada en el origen $n$.
\begin{equation*}
m=-\dfrac{\beta(b-1-r)}{p(b-1-\epsilon)}=-\dfrac{\beta p(b-1-\epsilon)}{(b-1-r)}\Longleftrightarrow (b-1-r)^2=(p(b-1-\epsilon))^2 \Longleftrightarrow
\end{equation*}
\begin{equation*}
\Longleftrightarrow (b-1-r)=p(b-1-\epsilon) \qquad \text{Con $b>1+r$}
\end{equation*}
\begin{equation*}
b(1-p)=1-p+r-p\epsilon \Longleftrightarrow b=1+\dfrac{r-p\epsilon}{1-p}
\end{equation*}
\begin{equation*}
b>1+r\qquad r>-p\epsilon
\end{equation*}
\begin{equation*}
\dfrac{r-p\epsilon}{1-p}>r\Longleftrightarrow r>\epsilon
\end{equation*}
$r>-p\epsilon$ es más restrictiva.\\

\begin{equation*}
n=-\dfrac{(\beta r+p\epsilon)}{p(b-1-\epsilon)}=-\dfrac{(r+\beta p\epsilon)}{(b-1-r)}\Longleftrightarrow (\beta r+p\epsilon)=(r+\beta p\epsilon) \Longleftrightarrow 
\end{equation*}
\begin{equation*}
\Longleftrightarrow \beta =1
\end{equation*}




\end{document}